\documentclass[a4paper]{article}

\usepackage[english]{babel}
\usepackage[utf8]{inputenc}
\usepackage{amsmath}
\usepackage{graphicx}
\usepackage{tgpagella}
\usepackage[T1]{fontenc}
\usepackage{hyperref}

\title{Lempel-Zev-Welch Compression Algorithm}

\author{Sasank Chilamkurthy, Tharun Kumar Reddy, Varun Bairaboina}

\date{25 April 2014}

\begin{document}
\maketitle

%\begin{abstract}
%We will implement two algorithms each from one of the two important aspects of information theory, which we have background on:
%Source Coding and channel coding.
%\end{abstract}


%
%\section{Overview}
%We will implement
%\begin {itemize}
%\item Lempel Zev Welch compression algorithm for compressing large files. This scheme was shown to asymptotically optimum for large files. We have achieved about 50 \% compression ratio.
%\item Decoding LDPC codes using Sum Product algorithm. These are capacity achieving codes proposed with very large parity check matrix.
%\end {itemize}





\section{Introduction}
This is a source coding algorithm or in other words, lossless compression algorithm. This is a dictionary based algorithm. It stores previously appeared string patterns in a dictionary. If a pattern in dictionary appears again, we transmit the index of the entry instead of the entire pattern. This is how compression is achieved. We implemented a very basic version of this algorithm. Many variants of lz algorithm are available which incorporate various optimisations in dictionary management etc.

\section{Encoding}

    \subsection{High Level Description}
        \begin{enumerate}
        \item Maintain a list of substrings appeared before in dictionary.
        \item Store the output string from the source in a buffer.
        \item Check if the present string at the buffer is there in the dictionary.
          \begin{itemize}
          \item If yes, then wait for one more symbol to come into the buffer and then go back to step 2.
          \item If no, then 
        		\begin{enumerate}
                \item Find the substring (buffer string excluding the last symbol) in the dictionary and transmit its index. 
                \item Transmit the last symbol. 
                \item Empty the buffer.
        \end {enumerate}
        \end{itemize}
        \end{enumerate}


    \subsection {Implementation details}
        \begin{itemize}
        \item Dictionary here requires searching by substrings. So we implemented dictionary as a \texttt{hashMap<String>} which has strings as its keys and integer indices as its values
        \item We will not transmit integer index in base 10 or 2 because that will be waste of character space as ascii characters can take values in range 0-127. So, we transmit index in base 128 system by converting it to corresponding character sequence.
        \end{itemize}


\section{Decoding}
Decoding is simmilar to encoding.

    \begin{enumerate}
    \item Maintain a list of substrings appeared before in dictionary.
    \item Store the input string in buffer. Fill buffer with incoming characters until it is of expected length. We calculate expected length from current dictionary size.
    \item Convert buffer[0:length-2] to integer using base 128 system. This was the index transmitted.
    \item Access string at this index from dictionary and append buffer[length-1] to it and return this.
    \end{enumerate}

\section{Analysis}
Everything (decoding and encoding) is done in constant time given performance of hashmap.

\end{document}